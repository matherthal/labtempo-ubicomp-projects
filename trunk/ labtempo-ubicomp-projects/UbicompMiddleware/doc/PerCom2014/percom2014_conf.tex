
%% bare_conf.tex
%% V1.3
%% 2007/01/11
%% by Michael Shell
%% See:
%% http://www.michaelshell.org/
%% for current contact information.
%%
%% This is a skeleton file demonstrating the use of IEEEtran.cls
%% (requires IEEEtran.cls version 1.7 or later) with an IEEE conference paper.
%%
%% Support sites:
%% http://www.michaelshell.org/tex/ieeetran/
%% http://www.ctan.org/tex-archive/macros/latex/contrib/IEEEtran/
%% and
%% http://www.ieee.org/

%%*************************************************************************
%% Legal Notice:
%% This code is offered as-is without any warranty either expressed or
%% implied; without even the implied warranty of MERCHANTABILITY or
%% FITNESS FOR A PARTICULAR PURPOSE! 
%% User assumes all risk.
%% In no event shall IEEE or any contributor to this code be liable for
%% any damages or losses, including, but not limited to, incidental,
%% consequential, or any other damages, resulting from the use or misuse
%% of any information contained here.
%%
%% All comments are the opinions of their respective authors and are not
%% necessarily endorsed by the IEEE.
%%
%% This work is distributed under the LaTeX Project Public License (LPPL)
%% ( http://www.latex-project.org/ ) version 1.3, and may be freely used,
%% distributed and modified. A copy of the LPPL, version 1.3, is included
%% in the base LaTeX documentation of all distributions of LaTeX released
%% 2003/12/01 or later.
%% Retain all contribution notices and credits.
%% ** Modified files should be clearly indicated as such, including  **
%% ** renaming them and changing author support contact information. **
%%
%% File list of work: IEEEtran.cls, IEEEtran_HOWTO.pdf, bare_adv.tex,
%%                    bare_conf.tex, bare_jrnl.tex, bare_jrnl_compsoc.tex
%%*************************************************************************

% *** Authors should verify (and, if needed, correct) their LaTeX system  ***
% *** with the testflow diagnostic prior to trusting their LaTeX platform ***
% *** with production work. IEEE's font choices can trigger bugs that do  ***
% *** not appear when using other class files.                            ***
% The testflow support page is at:
% http://www.michaelshell.org/tex/testflow/



% Note that the a4paper option is mainly intended so that authors in
% countries using A4 can easily print to A4 and see how their papers will
% look in print - the typesetting of the document will not typically be
% affected with changes in paper size (but the bottom and side margins will).
% Use the testflow package mentioned above to verify correct handling of
% both paper sizes by the user's LaTeX system.
%
% Also note that the "draftcls" or "draftclsnofoot", not "draft", option
% should be used if it is desired that the figures are to be displayed in
% draft mode.
%
\documentclass[conference]{IEEEtran}
% Add the compsoc option for Computer Society conferences.
%
% If IEEEtran.cls has not been installed into the LaTeX system files,
% manually specify the path to it like:
% \documentclass[conference]{../sty/IEEEtran}





% Some very useful LaTeX packages include:
% (uncomment the ones you want to load)


% *** MISC UTILITY PACKAGES ***
%
%\usepackage{ifpdf}
% Heiko Oberdiek's ifpdf.sty is very useful if you need conditional
% compilation based on whether the output is pdf or dvi.
% usage:
% \ifpdf
%   % pdf code
% \else
%   % dvi code
% \fi
% The latest version of ifpdf.sty can be obtained from:
% http://www.ctan.org/tex-archive/macros/latex/contrib/oberdiek/
% Also, note that IEEEtran.cls V1.7 and later provides a builtin
% \ifCLASSINFOpdf conditional that works the same way.
% When switching from latex to pdflatex and vice-versa, the compiler may
% have to be run twice to clear warning/error messages.






% *** CITATION PACKAGES ***
%
%\usepackage{cite}
% cite.sty was written by Donald Arseneau
% V1.6 and later of IEEEtran pre-defines the format of the cite.sty package
% \cite{} output to follow that of IEEE. Loading the cite package will
% result in citation numbers being automatically sorted and properly
% "compressed/ranged". e.g., [1], [9], [2], [7], [5], [6] without using
% cite.sty will become [1], [2], [5]--[7], [9] using cite.sty. cite.sty's
% \cite will automatically add leading space, if needed. Use cite.sty's
% noadjust option (cite.sty V3.8 and later) if you want to turn this off.
% cite.sty is already installed on most LaTeX systems. Be sure and use
% version 4.0 (2003-05-27) and later if using hyperref.sty. cite.sty does
% not currently provide for hyperlinked citations.
% The latest version can be obtained at:
% http://www.ctan.org/tex-archive/macros/latex/contrib/cite/
% The documentation is contained in the cite.sty file itself.






% *** GRAPHICS RELATED PACKAGES ***
%
\ifCLASSINFOpdf
  % \usepackage[pdftex]{graphicx}
  % declare the path(s) where your graphic files are
  % \graphicspath{{../pdf/}{../jpeg/}}
  % and their extensions so you won't have to specify these with
  % every instance of \includegraphics
  % \DeclareGraphicsExtensions{.pdf,.jpeg,.png}
\else
  % or other class option (dvipsone, dvipdf, if not using dvips). graphicx
  % will default to the driver specified in the system graphics.cfg if no
  % driver is specified.
  % \usepackage[dvips]{graphicx}
  % declare the path(s) where your graphic files are
  % \graphicspath{{../eps/}}
  % and their extensions so you won't have to specify these with
  % every instance of \includegraphics
  % \DeclareGraphicsExtensions{.eps}
\fi
% graphicx was written by David Carlisle and Sebastian Rahtz. It is
% required if you want graphics, photos, etc. graphicx.sty is already
% installed on most LaTeX systems. The latest version and documentation can
% be obtained at: 
% http://www.ctan.org/tex-archive/macros/latex/required/graphics/
% Another good source of documentation is "Using Imported Graphics in
% LaTeX2e" by Keith Reckdahl which can be found as epslatex.ps or
% epslatex.pdf at: http://www.ctan.org/tex-archive/info/
%
% latex, and pdflatex in dvi mode, support graphics in encapsulated
% postscript (.eps) format. pdflatex in pdf mode supports graphics
% in .pdf, .jpeg, .png and .mps (metapost) formats. Users should ensure
% that all non-photo figures use a vector format (.eps, .pdf, .mps) and
% not a bitmapped formats (.jpeg, .png). IEEE frowns on bitmapped formats
% which can result in "jaggedy"/blurry rendering of lines and letters as
% well as large increases in file sizes.
%
% You can find documentation about the pdfTeX application at:
% http://www.tug.org/applications/pdftex





% *** MATH PACKAGES ***
%
%\usepackage[cmex10]{amsmath}
% A popular package from the American Mathematical Society that provides
% many useful and powerful commands for dealing with mathematics. If using
% it, be sure to load this package with the cmex10 option to ensure that
% only type 1 fonts will utilized at all point sizes. Without this option,
% it is possible that some math symbols, particularly those within
% footnotes, will be rendered in bitmap form which will result in a
% document that can not be IEEE Xplore compliant!
%
% Also, note that the amsmath package sets \interdisplaylinepenalty to 10000
% thus preventing page breaks from occurring within multiline equations. Use:
%\interdisplaylinepenalty=2500
% after loading amsmath to restore such page breaks as IEEEtran.cls normally
% does. amsmath.sty is already installed on most LaTeX systems. The latest
% version and documentation can be obtained at:
% http://www.ctan.org/tex-archive/macros/latex/required/amslatex/math/





% *** SPECIALIZED LIST PACKAGES ***
%
%\usepackage{algorithmic}
% algorithmic.sty was written by Peter Williams and Rogerio Brito.
% This package provides an algorithmic environment fo describing algorithms.
% You can use the algorithmic environment in-text or within a figure
% environment to provide for a floating algorithm. Do NOT use the algorithm
% floating environment provided by algorithm.sty (by the same authors) or
% algorithm2e.sty (by Christophe Fiorio) as IEEE does not use dedicated
% algorithm float types and packages that provide these will not provide
% correct IEEE style captions. The latest version and documentation of
% algorithmic.sty can be obtained at:
% http://www.ctan.org/tex-archive/macros/latex/contrib/algorithms/
% There is also a support site at:
% http://algorithms.berlios.de/index.html
% Also of interest may be the (relatively newer and more customizable)
% algorithmicx.sty package by Szasz Janos:
% http://www.ctan.org/tex-archive/macros/latex/contrib/algorithmicx/




% *** ALIGNMENT PACKAGES ***
%
%\usepackage{array}
% Frank Mittelbach's and David Carlisle's array.sty patches and improves
% the standard LaTeX2e array and tabular environments to provide better
% appearance and additional user controls. As the default LaTeX2e table
% generation code is lacking to the point of almost being broken with
% respect to the quality of the end results, all users are strongly
% advised to use an enhanced (at the very least that provided by array.sty)
% set of table tools. array.sty is already installed on most systems. The
% latest version and documentation can be obtained at:
% http://www.ctan.org/tex-archive/macros/latex/required/tools/


%\usepackage{mdwmath}
%\usepackage{mdwtab}
% Also highly recommended is Mark Wooding's extremely powerful MDW tools,
% especially mdwmath.sty and mdwtab.sty which are used to format equations
% and tables, respectively. The MDWtools set is already installed on most
% LaTeX systems. The lastest version and documentation is available at:
% http://www.ctan.org/tex-archive/macros/latex/contrib/mdwtools/


% IEEEtran contains the IEEEeqnarray family of commands that can be used to
% generate multiline equations as well as matrices, tables, etc., of high
% quality.


%\usepackage{eqparbox}
% Also of notable interest is Scott Pakin's eqparbox package for creating
% (automatically sized) equal width boxes - aka "natural width parboxes".
% Available at:
% http://www.ctan.org/tex-archive/macros/latex/contrib/eqparbox/





% *** SUBFIGURE PACKAGES ***
%\usepackage[tight,footnotesize]{subfigure}
% subfigure.sty was written by Steven Douglas Cochran. This package makes it
% easy to put subfigures in your figures. e.g., "Figure 1a and 1b". For IEEE
% work, it is a good idea to load it with the tight package option to reduce
% the amount of white space around the subfigures. subfigure.sty is already
% installed on most LaTeX systems. The latest version and documentation can
% be obtained at:
% http://www.ctan.org/tex-archive/obsolete/macros/latex/contrib/subfigure/
% subfigure.sty has been superceeded by subfig.sty.



%\usepackage[caption=false]{caption}
%\usepackage[font=footnotesize]{subfig}
% subfig.sty, also written by Steven Douglas Cochran, is the modern
% replacement for subfigure.sty. However, subfig.sty requires and
% automatically loads Axel Sommerfeldt's caption.sty which will override
% IEEEtran.cls handling of captions and this will result in nonIEEE style
% figure/table captions. To prevent this problem, be sure and preload
% caption.sty with its "caption=false" package option. This is will preserve
% IEEEtran.cls handing of captions. Version 1.3 (2005/06/28) and later 
% (recommended due to many improvements over 1.2) of subfig.sty supports
% the caption=false option directly:
%\usepackage[caption=false,font=footnotesize]{subfig}
%
% The latest version and documentation can be obtained at:
% http://www.ctan.org/tex-archive/macros/latex/contrib/subfig/
% The latest version and documentation of caption.sty can be obtained at:
% http://www.ctan.org/tex-archive/macros/latex/contrib/caption/




% *** FLOAT PACKAGES ***
%
%\usepackage{fixltx2e}
% fixltx2e, the successor to the earlier fix2col.sty, was written by
% Frank Mittelbach and David Carlisle. This package corrects a few problems
% in the LaTeX2e kernel, the most notable of which is that in current
% LaTeX2e releases, the ordering of single and double column floats is not
% guaranteed to be preserved. Thus, an unpatched LaTeX2e can allow a
% single column figure to be placed prior to an earlier double column
% figure. The latest version and documentation can be found at:
% http://www.ctan.org/tex-archive/macros/latex/base/



%\usepackage{stfloats}
% stfloats.sty was written by Sigitas Tolusis. This package gives LaTeX2e
% the ability to do double column floats at the bottom of the page as well
% as the top. (e.g., "\begin{figure*}[!b]" is not normally possible in
% LaTeX2e). It also provides a command:
%\fnbelowfloat
% to enable the placement of footnotes below bottom floats (the standard
% LaTeX2e kernel puts them above bottom floats). This is an invasive package
% which rewrites many portions of the LaTeX2e float routines. It may not work
% with other packages that modify the LaTeX2e float routines. The latest
% version and documentation can be obtained at:
% http://www.ctan.org/tex-archive/macros/latex/contrib/sttools/
% Documentation is contained in the stfloats.sty comments as well as in the
% presfull.pdf file. Do not use the stfloats baselinefloat ability as IEEE
% does not allow \baselineskip to stretch. Authors submitting work to the
% IEEE should note that IEEE rarely uses double column equations and
% that authors should try to avoid such use. Do not be tempted to use the
% cuted.sty or midfloat.sty packages (also by Sigitas Tolusis) as IEEE does
% not format its papers in such ways.





% *** PDF, URL AND HYPERLINK PACKAGES ***
%
%\usepackage{url}
% url.sty was written by Donald Arseneau. It provides better support for
% handling and breaking URLs. url.sty is already installed on most LaTeX
% systems. The latest version can be obtained at:
% http://www.ctan.org/tex-archive/macros/latex/contrib/misc/
% Read the url.sty source comments for usage information. Basically,
% \url{my_url_here}.





% *** Do not adjust lengths that control margins, column widths, etc. ***
% *** Do not use packages that alter fonts (such as pslatex).         ***
% There should be no need to do such things with IEEEtran.cls V1.6 and later.
% (Unless specifically asked to do so by the journal or conference you plan
% to submit to, of course. )


% correct bad hyphenation here
\hyphenation{op-tical net-works semi-conduc-tor}


\begin{document}
%
% paper title
% can use linebreaks \\ within to get better formatting as desired
% \title{Interacting with the environment: a framework to develop ubiquitous applications for intelligent environments}
\title{Encapsulating the environment: a flexible framework to develop ubiquitous applications for intelligent environments}


% author names and affiliations
% use a multiple column layout for up to three different
% affiliations
\author{\IEEEauthorblockN{Matheus Erthal}
\IEEEauthorblockA{Institute of Computing\\
Federal Fluminense University\\
Rio de Janeiro, Brazil\\
Email: merthal@ic.uff.br}
\and
\IEEEauthorblockN{David Barreto}
\IEEEauthorblockA{Institute of Computing\\
Federal Fluminense University\\
Email: dbarreto@ic.uff.br}
\and
\IEEEauthorblockN{Douglas Mareli}
\IEEEauthorblockA{Institute of Computing\\
Federal Fluminense University\\
Email: dmareli@ic.uff.br}
\and
\IEEEauthorblockN{Orlando Loques}
\IEEEauthorblockA{Institute of Computing\\
Federal Fluminense University\\
Email: loques@ic.uff.br}}
% \author{\IEEEauthorblockN{Michael Shell}
% \IEEEauthorblockA{School of Electrical and\\Computer Engineering\\
% Georgia Institute of Technology\\
% Atlanta, Georgia 30332--0250\\
% Email: http://www.michaelshell.org/contact.html}
% \and
% \IEEEauthorblockN{Homer Simpson}
% \IEEEauthorblockA{Twentieth Century Fox\\
% Springfield, USA\\
% Email: homer@thesimpsons.com}
% \and
% \IEEEauthorblockN{James Kirk\\ and Montgomery Scott}
% \IEEEauthorblockA{Starfleet Academy\\
% San Francisco, California 96678-2391\\
% Telephone: (800) 555--1212\\
% Fax: (888) 555--1212}}

% conference papers do not typically use \thanks and this command
% is locked out in conference mode. If really needed, such as for
% the acknowledgment of grants, issue a \IEEEoverridecommandlockouts
% after \documentclass

% for over three affiliations, or if they all won't fit within the width
% of the page, use this alternative format:
% 
%\author{\IEEEauthorblockN{Michael Shell\IEEEauthorrefmark{1},
%Homer Simpson\IEEEauthorrefmark{2},
%James Kirk\IEEEauthorrefmark{3}, 
%Montgomery Scott\IEEEauthorrefmark{3} and
%Eldon Tyrell\IEEEauthorrefmark{4}}
%\IEEEauthorblockA{\IEEEauthorrefmark{1}School of Electrical and Computer Engineering\\
%Georgia Institute of Technology,
%Atlanta, Georgia 30332--0250\\ Email: see http://www.michaelshell.org/contact.html}
%\IEEEauthorblockA{\IEEEauthorrefmark{2}Twentieth Century Fox, Springfield, USA\\
%Email: homer@thesimpsons.com}
%\IEEEauthorblockA{\IEEEauthorrefmark{3}Starfleet Academy, San Francisco, California 96678-2391\\
%Telephone: (800) 555--1212, Fax: (888) 555--1212}
%\IEEEauthorblockA{\IEEEauthorrefmark{4}Tyrell Inc., 123 Replicant Street, Los Angeles, California 90210--4321}}




% use for special paper notices
%\IEEEspecialpapernotice{(Invited Paper)}




% make the title area
\maketitle


\begin{abstract}
%\boldmath
Due to recent advances in mobile computing and wireless communication technologies, we can see the emergence of a favorable scenario for building ubiquitous/pervasive applications. This work proposes a new framework that aims at building those applications, providing a set of concepts and implementation tools. We propose abstractions that allow developers to handle the resources spread in the environment in a simple and homogeneous way, and to interpret context information. In order to demonstrate the feasibility of the proposal, the concepts were implemented in a platform named \textit{SmartAndroid}. A ubiquitous applications prototyping interface was implemented over this platform to allow testing of different environment configurations before purchasing all devices; and also a context rules composition interface, where end users can define their preferences in the environment.

% Context awareness is reaching increasingly visibility in the current applications development scenario. The goal of this paper is to propose a new solution for representing, distributing and interpreting context informations, thus allowing the construction of applications for intelligent ambients. The infrastructure described in this paper exposes the ambient resources in a higher level, leading to a simpler and more dynamic creation of context rules; and by means of a graphical interface, end users are able to easily create and customize rules in the ambient. The proposed concepts have been implemented as a platform focused on building smart homes.
% 
% This article describes the Pervasive Applications Prototyping and Management Interface (IPGAP) that aims to provide a platform to support construction, test and execution of applications for smart ambients. In order to provide these features capabilities, our tool helps to perform simulation of sensors and actuators as well as means to visualize the interaction of real components which are inside the ambient. This way the developer will be able to construct applications without having a complete smart ambient infrastructure.

\end{abstract}
% IEEEtran.cls defaults to using nonbold math in the Abstract.
% This preserves the distinction between vectors and scalars. However,
% if the conference you are submitting to favors bold math in the abstract,
% then you can use LaTeX's standard command \boldmath at the very start
% of the abstract to achieve this. Many IEEE journals/conferences frown on
% math in the abstract anyway.

% no keywords




% For peer review papers, you can put extra information on the cover
% page as needed:
% \ifCLASSOPTIONpeerreview
% \begin{center} \bfseries EDICS Category: 3-BBND \end{center}
% \fi
%
% For peerreview papers, this IEEEtran command inserts a page break and
% creates the second title. It will be ignored for other modes.
\IEEEpeerreviewmaketitle



\section{Introduction}
% no \IEEEPARstart
% You must have at least 2 lines in the paragraph with the drop letter
% (should never be an issue)
The field of Ubiquitous Computing (UbiComp), nowadays widely discussed, was first proposed by Mark Weiser in the 1990's~\cite{Weiser1991century}. Also referred by pervasive computing, the field aims at providing a different paradigm of human-machine interaction. As traditional applications provides services to users through their explicit interaction (using devices as mouse, keyword, monitor, for example), in ubiquitous applications the interaction happens without the need of explicit interaction, i.e. the application tries to discover the users needs through the acquisition of context (using sensors) and the knowledge of their preferences, and providing services in the environment (using actuators). These environments enhanced with sensors and actuators to provide automated services to persons are also called Intelligent Environments (IE).

% The construction of a ubiquitous application from the start is not a simple task, since the developer will have concerns from the communication level to the representation of resources in high level before even start programming the business logic of her application. 
The construction and manipulation of ubiquitous applications represent major challenges for developers, especially in terms of technical knowledge required and the availability of real devices during application development. Some of these challenges can be well highlighted: (i) there are difficulties in establishing a common protocol for communication between the components of the distributed system, because of the \textit{heterogeneity of devices} involved, (ii) the interactivity of ubiquitous applications is hampered depending on the amount and \textit{variety of context information and services} available in the environment, (iii) \textit{developing and testing applications} require high availability of resources, such as sensors (e.g. presence, lighting, temperature), actuators (e.g. keys, alarms, smart-tvs), including new embedded devices, or physical spaces, such as a house for applications of type smart home.

% From the Mark Weiser proposed in the 1990s, researchers in the field of ubicomp / pervasive have proposed changes in human-machine interaction, in order to make the use of devices increasingly transparent environment. This enables the user to focus on the task at hand and not the tool to do it. From these ideas came the concept of smart environments, where sensors and actuators interconnected network are able to provide relevant information about the environment to applications and users, and effectively act in this environment and change its state.



%FIXME: araujo � um artigo em PT, arrumar um em EN
In this work we propose a framework for developing ubiquitous applications in IE. The goal is to provide support for the programming, testing and execution of applications, thus allowing to deal consistently with systems great complexity. The framework stands out for addressing the challenges already identified in UbiComp~\cite{araujo2003}. The \textit{heterogeneity of devices} is handled through the definition of a Distributed Component Model, that provides abstractions to encapsulate these devices, also called resources, enabling developers to interact with them seamlessly. Regarding the \textit{variety of context information and services} issue, we propose a solution for context interpretation that allows developers to create and manage context rules at runtime, and users to set their preferences in the IE. Finally, regarding the issue on the \textit{resource availability}, the framework includes an application interface for the prototyping and management of pervasive applications (IPGAP), focused on visualization and testing ubiquitous applications, mixing real and virtual components.
%in which the basic component has a uniform structure defined as a Resource Agent (RA), formerly proposed in~\cite{cardoso2006} as an entity gathering context information. In this work, the RA, while maintaining context information, acts as a component that encapsulates the device code associated with it, including aspects of interaction with application components. 
%The \textit{heterogeneity of devices} is handled through the definition of a Distributed Component Model, in which the basic component has a uniform structure defined as a Resource Agent (RA), formerly proposed in~\cite{cardoso2006} as an entity gathering context information. In this work, the RA, while maintaining context information, acts as a component that encapsulates the device code associated with it, including aspects of interaction with application components. 

The concepts of the framework were materialized on a platform called \textit{SmartAndroid}\footnote{www.tempo.uff.br/smartandroid}, whose development has enabled appraisal in order to prove that the conceptual framework facilitates the process of building ubiquitous applications. We implemented some use cases to validate different aspects of the framework proposed and to testify its capabilities, supported by the elaboration of competency questions, which is a mechanism mostly used to evaluate ontologies, but can also be used as an assessment to this work. 
%power of representation 
%testify the feasibility 
% has enabled the construction of use cases that demonstrate the effectiveness of the approach to provide context awareness to new applications.

% The concepts of the framework have been implemented on a platform called \textit{SmartAndroid}\footnote{www.tempo.uff.br/smartandroid} developed in the context of the project. The implementation of this project has enabled appraisal in order to prove that the conceptual framework facilitates the process of building ubiquitous applications.  One of the evaluations occurred during the processing of an application running on a purely local ubiquitous application. In another assessment, an application was built to verify the feasibility of implementation of Context Model, through the exploitation of communication mechanisms used in the Distributed Component Model.


The remainder of this paper is organized as follows. First we present in Section~\ref{sec:overview} an overview of the main concepts used as a basis for the framework's development. In Section~\ref{sec:framework}, we present the framework's overall architecture and main features. We then, in Section~\ref{sec:evaluation}, present a proof of concept demonstrating the feasibility of building ubiquitous applications using the framework and we show examples of applications that explores the key features of the IE. Finally, in Section~\ref{sec:related} we compare our approach with related work and in Section~\ref{sec:conclusion} we conclude this paper with the main remarks.


\section{Related Work} \label{sec:related}
% [ubicomp is nice but it is not priceless]
The benefits to users and developers that UbiComp foresees reach beyond the horizon of many imaginative researchers. The possibilities that arise from it surpass the bounds established by keyboards and mice, our standard interfaces. Nonetheless, building such adaptable applications requires much effort from developers that see themselves immersed in a universe of device's specifications and communication technologies, apart from the problem itself that the application must solve.

%ubicomp implies dealing with limited and dinamically varying computational resources [kindberg2002ieee]

Therefore, many researchers have focused on diminishing those obstacles by providing abstractions, middlewares, services, tools, and other suportive techniques, not only for development but also for prototyping.

As one of the most famous, the Gaia middleware is designed to facilitate the construction of applications for IE. It consists of a set of core services and a framework for building distributed context aware applications. 
This project 
%uses operating systems abstractions to define its architecture
%have many goals that include the provision of mechanisms of acquisition of context; composition of rules (using first logic order), 
%%maintenance of hardwares and softwares description and about applications; 
%search mechanism to find resources
%presence and execution status of resources
%interpretation mechanism that gathers and interprets informations, as defined by programmers
%mvc
%The main goals of Gaia's approach are the acquisition of context by applications, the monitoring of entities location, the maintenance of hardwares and softwares descriptions, and 
%FIXME: second phrase was copied
%Active Spaces
%ontologies

% [many researchers research but few people use]

% [compare based on flexibility]

% [compare based on user/developer center focus]

% [compare based on pontual solution]

% [talk about market, which approach are already in use?] Google IS, onX

% [explain that our solution strives to accomodate several problems of UbiComp and to be flexible enough to accomodate the others]

\section{Overview} \label{sec:overview}
% [context/contx sensibility]
It is easy to notice that the context information is a first concern topic in ubiquitous systems. But what is ``context''? Many authors have proposed definitions to this concept, though most are little accurate~\cite{baldauf2007survey}. Synthetizing previous definitions, Dey and Abowd have proposed that context is ``any information that can be used to characterize the situation of entities (i.e., whether a person, place or object) that are considered relevant to the interaction between a user and an application, including the user and the application themselves''~\cite{Dey2001}. In a IE ``Places'' are the rooms, the floors of a building, or spaces in general that enable the localization of other entities, ``People'' are individuals that populate the environment and interact with it, and ``things'' are virtual representations of physical objects or software components.
%Considering the diversity of devices, systems and protocols, one 
%Many authors have proposed definitions to this concept, though most use examples to explain it or adopt a restrictive definition, that may not contemplate a set of systems. <- wrong. Most are wide or use synonyms
%Synthetizing previous definitions, Dey and Abowd have proposed that context is any relevant information used to caracterize the situation of entities, specifically: people, places and things.

Context aware (or context sensible) applications is a designation to those applications that not only are capable of knowing the context of the environment, but also react to it either by means of changing the environment or in a software level.

[distribs sys]




\section{Framework} \label{sec:framework}
[framework description] 

\subsection{Resource Agents}

[communication mechanisms]

\subsection{Architeture}
[distributed components model]

\subsection{Management Services}
[resource register service]

[resource discovery service]

[resource location service]

[location concerns]

[security concerns]

\subsection{Context Interpretation}
[directly programmed]

[as a separated entity]

\subsection{Framework Implementation}
[android and other techs]

[impl of AR and services]

[impl of context rules]

[context rules conflicts is an ongoing study that aims at discover a conflict in runtime, ]

\subsection{Management and Prototyping Interface}
% interface de prototipagem e gerenciamento de aplica��es pervasivas
% prototyping and management of pervasive applications interface
% pervasive applications' management and prototyping interface
% PAMPI
% management and prototyping interface MPI
[description]

[main figure]

[rules composer]


\section{Case Study} \label{sec:evaluation}


\section{Conclusion} \label{sec:conclusion}
The conclusion goes here.




% use section* for acknowledgement
\section*{Acknowledgment}


The authors would like to thank...


% references section
\bibliographystyle{IEEEtran}
\bibliography{references}




% that's all folks
\end{document}
