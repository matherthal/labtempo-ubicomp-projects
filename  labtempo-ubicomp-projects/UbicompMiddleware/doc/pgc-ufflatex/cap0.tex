% --- -----------------------------------------------------------------
% --- Elementos usados na Capa e na Folha de Rosto.
% --- EXPRESS�ES ENTRE <> DEVER�O SER COMPLETADAS COM A INFORMA��O ESPEC�FICA DO TRABALHO
% --- E OS S�MBOLOS <> DEVEM SER RETIRADOS 
% --- -----------------------------------------------------------------
\autor{<NOME DO ALUNO>} % deve ser escrito em maiusculo

\titulo{<T\'ITULO DO TRABALHO>}

\instituicao{UNIVERSIDADE FEDERAL FLUMINENSE}

\orientador{<NOME DO ORIENTADOR>}

\coorientador{<NOME DO CO-ORIENTADOR>} % se nao existir co-orientador apague essa linha

\local{NITER\'{O}I}

\data{<ANO>} % ano da defesa

\comentario{<Tese de Doutorado OU Disserta��o de Mestrado> apresentada ao Programa de P\'{o}s-Gradua\c{c}\~{a}o em Computa\c{c}\~{a}o da \mbox{Universidade} Federal Fluminense como requisito parcial para a obten\c{c}\~{a}o do Grau de \mbox{<Doutor ou Mestre> em Computa\c{c}\~{a}o}. \'{A}rea de concentra\c{c}\~{a}o: \mbox{<\'AREA DE CONCENTRA\c{C}\~AO.>}} %preencha com a sua area de concentracao


% --- -----------------------------------------------------------------
% --- Capa. (Capa externa, aquela com as letrinhas douradas)(Obrigatorio)
% --- ----------------------------------------------------------------
\capa

% --- -----------------------------------------------------------------
% --- Folha de rosto. (Obrigatorio)
% --- ----------------------------------------------------------------
\folhaderosto


\pagestyle{ruledheader}
\setcounter{page}{1}
\pagenumbering{roman}

% --- -----------------------------------------------------------------
% --- Termo de aprovacao. (Obrigatorio)
% --- ----------------------------------------------------------------
\cleardoublepage
\thispagestyle{empty}

\vspace{-60mm}

\begin{center}
   {\large <NOME DO ALUNO>}\\
   \vspace{7mm}

   <T\'ITULO DO TRABALHO>\\
  \vspace{10mm}
\end{center}

\noindent
\begin{flushright}
\begin{minipage}[t]{8cm}

<Tese de Doutorado ou Disserta��o de Mestrado> apresentada ao Programa de P\'{o}s-Gradua\c{c}\~{a}o em Computa\c{c}\~{a}o da Universidade Federal Fluminense como requisito parcial para a obten\c{c}\~{a}o do \mbox{Grau} de <Doutor ou Mestre> em Computa\c{c}\~{a}o. \'{A}rea de concentra\c{c}\~{a}o: \mbox{<\'AREA DE CONCENTRA\c{C}\~AO.>} %preencha com a sua area de concentracao

\end{minipage}
\end{flushright}
\vspace{1.0 cm}
\noindent
Aprovada em <MES> de <ANO>. \\
\begin{flushright}
  \parbox{11cm}
  {
  \begin{center}
  BANCA EXAMINADORA \\
  \vspace{6mm}
  \rule{11cm}{.1mm} \\
    Prof. <NOME do ORIENTADOR> - Orientador, UFF \\
    \vspace{6mm}
  \rule{11cm}{.1mm} \\
    Prof. <NOME DO AVALIADOR>, <INSTITUI\c{C}\~AO>\\
    \vspace{6mm}
  \rule{11cm}{.1mm} \\
    Prof. <NOME DO AVALIADOR>, <INSTITUI\c{C}\~AO>\\
  \vspace{4mm}
  \rule{11cm}{.1mm} \\
    Prof. <NOME DO AVALIADOR>, <INSTITUI\c{C}\~AO>\\
    \vspace{6mm}
  \rule{11cm}{.1mm} \\
    Prof. <NOME DO AVALIADOR>, <INSTITUI\c{C}\~AO>\\
  \vspace{6mm}
  \end{center}
  }
\end{flushright}
\begin{center}
  \vspace{4mm}
  Niter\'{o}i \\
  %\vspace{6mm}
  <ANO>

\end{center}

% --- -----------------------------------------------------------------
% --- Dedicatoria.(Opcional)
% --- -----------------------------------------------------------------
\cleardoublepage
\thispagestyle{empty}
\vspace*{200mm}

\begin{flushright}
{\em 
Dedicat�ria(s): Elemento opcional onde o autor presta homenagem ou dedica seu trabalho (ABNT, 2005).
}
\end{flushright}
\newpage


% --- -----------------------------------------------------------------
% --- Agradecimentos.(Opcional)
% --- -----------------------------------------------------------------
\pretextualchapter{Agradecimentos}
\hspace{5mm}
Elemento opcional, colocado ap�s a dedicat�ria (ABNT, 2005). 

% --- -----------------------------------------------------------------
% --- Resumo em portugues.(Obrigatorio)
% --- -----------------------------------------------------------------
\begin{resumo}

Elemento obrigat�rio, constitu�do de uma sequ�ncia de frases concisas e objetivas e n�o de uma simples enumera��o de t�picos, n�o ultrapassando 500 palavras (ABNT, 2005).

{\hspace{-8mm} \bf{Palavras-chave}}: Palavras representativas do conte�do do trabalho, isto �, palavras-chave e/ou descritores, conforme a ABNT NBR 6028 (ABNT, 2005).

\end{resumo}

% --- -----------------------------------------------------------------
% --- Resumo em lingua estrangeira.(Obrigatorio)
% --- -----------------------------------------------------------------
\begin{abstract}

Elemento obrigat�rio, em l�ngua estrangeira, com as mesmas caracter�sticas do resumo em l�ngua vern�cula (ABNT, 2005).

{\hspace{-8mm} \bf{Keywords}}: Palavras representativas do conte�do do trabalho, isto �, palavras-chave e/ou descritores, na l�ngua (ABNT, 2005).

\end{abstract}

% --- -----------------------------------------------------------------
% --- Lista de figuras.(Opcional)
% --- -----------------------------------------------------------------
%\cleardoublepage
\listoffigures


% --- -----------------------------------------------------------------
% --- Lista de tabelas.(Opcional)
% --- -----------------------------------------------------------------
\cleardoublepage
%\label{pag:last_page_introduction}
\listoftables
\cleardoublepage

% --- -----------------------------------------------------------------
% --- Lista de abreviatura.(Opcional)
%Elemento opcional, que consiste na rela��o alfab�tica das abreviaturas e siglas utilizadas no texto, seguidas das %palavras ou express�es correspondentes grafadas por extenso. Recomenda-se a elabora��o de lista pr�pria para cada %tipo (ABNT, 2005).
% --- ----------------------------------------------------------------
\cleardoublepage
\pretextualchapter{Lista de Abreviaturas e Siglas}
\begin{tabular}{lcl}
<ABREVIATURA> & : & <SIGNIFICADO>;\\
<ABREVIATURA> & : & <SIGNIFICADO>;\\
<ABREVIATURA> & : & <SIGNIFICADO>;\\
\end{tabular}
% --- -----------------------------------------------------------------
% --- Sumario.(Obrigatorio)
% --- -----------------------------------------------------------------
\pagestyle{ruledheader}
\tableofcontents


